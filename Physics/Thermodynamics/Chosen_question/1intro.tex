\section {Недостаточность классической теории теплоемкостей}

\subsection {Сравнение классической теории с опытом}

Во многом, классическая теория дает правильные результаты (для определенного круга явлений). \\

Однако многие явления она не объясняет. Ряд опытных фактов находится в противоречии с этой теорией. \\

Прежде всего, классическая теория не дает объяснения зависимости теплоемкости тел от температуры. В таблице 1 приведены для примера молярные теплоемкости газообразного водорода при разных температурах.

\begin{table} [h!]
    \begin{center}
        \caption*{Таблца 1}
        \begin{tabular} {| P {5cm} | P {5cm} |}
            \hline

            Т, К  & $ C_V, \frac {\texttt {кал}} {\texttt {моль} \cdot \texttt {К}} $ \\ [0.1cm]
            \hline

            35   & 2.98 \\
            \hline
            100  & 3.10 \\
            \hline
            290  & 4.90 \\
            \hline
            600  & 5.08 \\
            \hline
            800  & 5.22 \\
            \hline
            1000 & 5.36 \\
            \hline

        \end{tabular}
    \end{center}
\end{table}

Можно было бы попытаться объяснить зависимость теплоемкости от температуры негармоничностью колебательных степеней свободы при больших амплитудах колебаний, однако эти соображения теряют силу при низких температурах, где расхождения с теорией видны особенно резко.
Кроме того, следует также иметь ввиду эксперементально установленный факт, что: $ \lim_{T\to0} C_V, C_P = 0 $.

\subsection {Непоследовательность классической теории}

По теореме о равномерном распределении энергии все степени свободы равноправны. Поэтому достаточно лишь подсичтать полное число степеней свободы, не обращая внимания на их природу. Однако, классическая теория по каким-то причинам учитывает одни и отбрасывает другие степени свободы. \\ [0.2cm]
Так, атом одноатомного газа рассматривается как материальная точка с тремя степенями свободы. Но атом - не точка, более того, атом обладает сложной внутренней структурой, вследствии чего обладает числом степеней свободы больше шести. Тогда, если посмотреть на атом с этой точки зрения, согласно классической теории, его $ C_V $ должна быть много больше 6 $ \frac {\texttt {кал}} {\texttt {моль} \cdot \texttt {К}} \, , $ но это противоречит фактам.

\subsection {Выводы}

Таким образом, опытные факты приводят к заключению, что эффективный вклад вносят только некоторые степени свободы. При понижении температуры некоторые степени свободы становятся мало эффективными и, наконец, совсем перестают вносить свой вклад. В таких случаях говорят, что степени свободы "заморожены". \\ [0.2cm]

Вышеописанные трудности были преодолены после того, как теория теплоемкости была построена на квантовой основе.

\newpage