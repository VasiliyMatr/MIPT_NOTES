\section {Квантовая теория теплоемкостей}

\subsection {Квантовые законы и простейшие следствия из них}

Как доказывается в квантовой механике, внутренняя энергия атомных систем может принимать лишь дискретные значения. \\ [0.2cm]

Двухатомная моллекула, когда речь идет о малых колебаниях ее атомов, может рассматриваться как \textbf {гармонический осциллятор}. Возможные значения колебательной энергии такой системы представляются формулой:

\begin{equation}
    \epsilon_n = \left( n + \frac {1}{2} \right) h\nu,
    \label{oscEn}
\end{equation}

где $ \nu $ - частота осциллятора, $ n $ - целое неотрицательное число, $ h $ - постоянная планка. \\ [0.2cm]

Из формулы видно, что самому низкому уровню энергии соответствует энергия $ \epsilon_0 = \frac {1}{2} h\nu $. Такие колебания называются нулевыми. Ближайшим к $ \epsilon_0 $ является энергетический уровень, отстоящий на $ \Delta\epsilon = h\nu $ \\ [0.2cm]

Допустим теперь, что газ состоит из гармонических осцилляторов (например - двухатомных молекул). Если температура газа достаточно мала, чтобы выполнялось соотношение: $ kT \leq h\nu, $ то средняя энергия теплового движения будет порядка $ kT $. А значит, такой энергии недостаточно, чтобы возбудить осциллятор, т.е. перевести его с нулевого уровня на ближайший энергетический уровень. Возбуждение может произойти только при столкновении с моллекулой, энергия которой значительно больше средней. Однако, таких моллекул относительно мало, так что \textit {практически все осцилляторы останутся на нижнем энергетическом уровне}. \\
Такая картина будет сохраняться в некотором диапазоне температур, следовательно, в этом диапазоне колебательная энергия осцилляторов практически не зависит от температуры, а следовательно, эта энергия не влияет на теплоемкость. \\
Т.о. \textit {классическая модель применима только в случае, если выполняется соотношение: $ kT \leq h\nu $}. Температура

\begin{equation}
    T_v = \frac {h\nu}{k}
    \label{TvOsc}
\end{equation}

называется \textbf {характерестической температурой}. при $ T \geq T_v $ колебания осцилляторов существенно влияют на теплоемкость двухатомного газа. \\ [0.2cm]

Анологично влияет на теплоемкость газов и вращение молекул. Энергия вращения также квантуется. Ее возможные значения по квантовой механике определяются формулой:

\begin{equation}
    \epsilon_l = \frac {h^2}{8\pi^2I}l(l + 1),
    \label{rotEn}
\end{equation}

где $ I $ - момент инерции молекулы, а $ l $ - целое неотрицательное число. при $ l = 0 $ вращения не возбуждены. Характеристическая температура для вращения молекул определяется формулой:

\begin{equation}
    T_r = \frac {\epsilon_1}{k} = \frac {h^2}{4\pi^2Ik}
    \label{TvRot}
\end{equation}

\textit {При $ T >> T_r $ справедлива классическая теория, при $ T << T_r $ вращения не возбуждены и не оказывают влияния на теплоемкость. По той же причине не возбуждены вращения атомов одноатомарных газов.} \\ [0.2cm]

\newpage

Вышеописанные соображения применимы к любым квантовым системам. Они показывают, что \textit {дискретность энергитических уровней несовместима с классической теоремой о равномерном распределении энергии по степеням свободы}. Только при энергии теплового движения $ kT $ много больше разности энергий между высшими и низшими энергитическими уровнями, система ведет себя как классическая. Т.е. \textit {чем выше температура, тем лучше оправдывается классическая теорема о равномерном распределении энергии по степеням свободы}.

\subsection {Применение распределения Больцмана}

Будем представлять тело, как систему $ N $ молекул, колебания которых не связаны друг с другом. Применим к ней закон распределения Больцмана, предпологая, что энергитические уровни дискретны:

\begin{equation}
    \overline\epsilon = \frac {1}{Z} \displaystyle\int_{}^{}\epsilon e^{-\alpha\epsilon} d\Gamma = - \frac {1}{Z} \frac {dZ}{d\alpha},
    \label{Bolc1}
\end{equation}

где $ \alpha = \frac {1}{kT} $, $ d\Gamma $ - элемент пространства скоростей, а $ Z $ определяется условием нормировки:

\begin{equation}
    Z = \displaystyle\int_{}^{}e^{-\alpha\epsilon} d\Gamma
    \label{Bolc2}
\end{equation}

Средняя энергия, приходящаяся на одну молекулу в состоянии термодинамического равновесия, определяется уравнением:

\begin{equation}
    \overline\epsilon = \frac {1}{N} \displaystyle\sum_{i = 0}^{\infty}N_i\epsilon_i
    \label{avgEn1}
\end{equation}

С учетом распределения Больцмана и условия нормировки получим:

\begin{equation}
    \overline\epsilon = \frac {\displaystyle\sum_{i = 0}^{\infty}\epsilon_i e^{-\alpha\epsilon_i}} {\displaystyle\sum_{i = 0}^{\infty} e^{-\alpha\epsilon_i}},
    \label{avgEn2}
\end{equation}

или

\begin{equation}
    \overline\epsilon = - \frac {1}{Z} \frac {dZ}{d\alpha} = - \frac {d}{d\alpha} \left( ln Z \right),
    \label{avgEn3}
\end{equation}

где введено обозначение

\begin{equation}
    Z = \displaystyle\sum_{i = 0}^{\infty} e^{-\alpha\epsilon_i} = \displaystyle\sum_{i = 0}^{\infty} e^{-\frac {\epsilon_i}{kT}}
    \label{avgEn4}
\end{equation}

Выражение (\ref {avgEn4}) называется \textbf {статистической суммой} или \textbf {суммой состояний}. \\ [0.2cm]

Рассмотрим далее систему одномерных гармонических осцилляторов. Уровни энергии гармонического осциллятора определяются формулой (\ref {oscEn}). Тогда для суммы состояний получаем:

\begin{equation}
    Z = e^{\left( \frac {-\alpha h\nu}{2} \right)} \displaystyle\sum_{i = 0}^{\infty} e^{-i\alpha h\nu} = \frac {e^{\left( \frac {-\alpha h\nu}{2} \right)}}{1 - e^{-\alpha h\nu}},
    \label{condSum}
\end{equation}

а для средней энергии осциллятора:

\begin{equation}
    \overline\epsilon = - \frac {d}{d\alpha} \left( ln Z \right) = \frac {h\nu}{2} + \frac {h\nu}{e^{\left( \frac {h\nu}{kT} \right)} - 1}
    \label{quAvgEn1}
\end{equation}

\newpage

Слагаемое $ \frac {h\nu}{2} $ есть \textbf {нулевая энергия} гармонического осциллятора. Она не зависит от температуры и \linebreak не имеет отношения к тепловому движению. В теории теплоемкости ее можно опустить, тогда получим:

\begin{equation}
    \overline\epsilon = \frac {h\nu}{e^{\left( \frac {h\nu}{kT} \right)} - 1}
    \label{quAvgEn2}
\end{equation}

Согласно этой формуле, елси $ h\nu << kT $, что имеет место при высоких температурах, то $ e^{\left( \frac {h\nu}{kT} \right)} \approx 1 + \frac {h\nu}{kT}$. В этом приближении формула (\ref {quAvgEn2}) переходит в классическую: $ \overline\epsilon = kT $.

\subsection {Квантовая теория теплоемкости Эйнштейна}

Формула (\ref {quAvgEn2}) была положена Эйнштейном в основу квантовой теории теплоемкости твердых тел. Он пользовался той же моделью твердого тела, какая применялась в классической теории. Атомы кристаллической решетки рассматривались как гармонические осцилляторы, совершающие тепловые колебания около положений равновесия с одной и той же частотой $ \nu $. Осцилляторы считаются трехмерными. На каждую степень свободы приходится средняя энергия  тепловых колебанй $ \overline\epsilon, $ а на каждый атом - $ 3\overline\epsilon $. Тогда внутренняя энергия одного моля определяется выражением:

\begin{equation}
    U = 3N_A\overline\epsilon = \frac {3N_Ah\nu}{e^{\left( \frac {h\nu}{kT} \right)} - 1}
    \label{molEn}
\end{equation}

Тогда можно получить выражение для молярной теплоемкости кристаллической решетки твердых тел:

\begin{equation}
    C_V = \frac {dU}{dT} = \frac {3R \left( \frac {h\nu}{kT} \right) ^2}{\left( e^{\left( \frac {h\nu}{kT} \right)} - 1 \right) ^2} \cdot e^{\left( \frac {h\nu}{kT} \right)} - \texttt {\bf формула Эйнштейна}
    \label{Einstain}
\end{equation}

При высоких температурах, когда $ \frac {h\nu}{kT} << 1, $ она переходит в классическую формулу:

\begin{equation}
    C_V = 3R
    \label{classicCV}
\end{equation}

В другом предельном случае низких температур, когда $ \frac {h\nu}{kT} >> 1, $ можно пренебречь единицей в знаменателе и получить:

\begin{equation}
    C_V = 3R\left( \frac {h\nu}{kT} \right)^2 \cdot e^{-\left( \frac {h\nu}{kT} \right)}
    \label{lowTCV}
\end{equation}

При $ T \to 0 $ выражение (\ref {lowTCV}) стремится к 0, что согласуется с теоремой Нернста.

\subsection {Недостатки квантовой теории тепоемкости Эйнштейна}

\textit {Несмотря на описанные выше преимущества квантовой теории тепоемкости Эйнштейна, она дает только качественное согласие с опытом.} Например, при $ T \to 0 $, $ C_V \to 0 $ слишком быстро (практически экспоненциально). Опыт же показывает, что в действительности зависимость степенная. При остальных температурах, у данной формулы есть те же недостатки. Однако, это связано не с существом квантовой теории, а с упрощением расчета, в котором предпологается, что все гармонические осцилляторы колеблются с одной и той же частотой. \\ [0.2cm]

На самом деле кристаллическую решетку следует рассматривать как связанную систему \linebreak взаимодействующих частиц. Малые колебания такой системы получаются в результате наложения многих гармонических колебаний с различными частотами. Число частот очень велико - порядка числа степеней свободы системы. При вычислении теплоемкости тело можно рассматривать как систему гармонических осцилляторов, но с различными частотами. Тогда задача сводится к вычислению этих частот, т.е. к отысканию так называемого \textbf {спектра частот}. На это указывал еще сам Эйнштейн. \\ [0.2cm]

Задача о спектре частот кристаллической решетки твердого тела рассматривалась Дебаем, а затем Борном и Карманом. \\ [0.2cm]

\textit {Теория Эйнштейна, разумеется, применима и к колебательной теплоемкости двухатомных или многоатомных газов.} Аналогично, можно построить и теорию вращательной теплоемкости. Однако, вычисления будут сложнее из-за более сложной структуры энергетического спектра.