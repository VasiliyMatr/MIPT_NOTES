\section {Квантовая теория теплоемкостей}

\subsection {Квантовые законы и простейшие следствия из них}

Как доказывается в квантовой механике, внутренняя энергия атомных систем может принимать лишь дискретные значения. \\ [0.2cm]

Двухатомная моллекула, когда речь идет о малых колебаниях ее атомов, может рассматриваться как гармонический осциллятор. Возможные значения колебательной энергии такой системы представляются формулой:

\begin{equation}
    \epsilon_n = \left( n + \frac {1}{2} \right) h\nu,
    \label{oscEn}
\end{equation}

где $ \nu $ - частота осциллятора, $ n $ - целое неотрицательное число, $ h $ - постоянная планка. \\ [0.2cm]

Из формулы видно, что самому низкому уровню энергии соответствует энергия $ \epsilon_0 = \frac {1}{2} h\nu $. Такие колебания называются нулевыми. Ближайшим к $ \epsilon_0 $ является энергетический уровень, отстоящий на $ \Delta\epsilon = h\nu $ \\ [0.2cm]

Допустим теперь, что газ состоит из гармонических осцилляторов (например - двухатомных молекул). Если температура газа достаточно мала, чтобы выполнялось соотношение: $ kT \leq h\nu, $ то средняя энергия теплового движения будет порядка $ kT $. А значит, такой энергии недостаточно, чтобы возбудить осциллятор, т.е. перевести его с нулевого уровня на ближайший энергетический уровень. Возбуждение может произойти только при столкновении с моллекулой, энергия которой значительно больше средней. Однако, таких моллекул относительно мало, так что практически все осцилляторы останутся на нижнем энергетическом уровне. \\
Такая картина будет сохраняться в некотором диапазоне температур, следовательно, в этом диапазоне колебательная энергия осцилляторов практически не зависит от температуры, а следовательно, эта энергия не влияет на теплоемкость. \\
Т.о. классическая модель применима только в случае, если выполняется соотношение: $ kT \leq h\nu $. Температура

\begin{equation}
    T_v = \frac {h\nu}{k}
    \label{TV}
\end{equation}

называется характерестической температурой. при $ T \geq T_v $ колебания осцилляторов существенно влияют на теплоемкость двухатомного газа. \\ [0.2cm]

Анологично влияет на теплоемкость газов и вращение молекул. Энергия вращения также квантуется. Ее возможные значения по квантовой механике определяются формулой:

\begin{equation}
    \epsilon_l = \frac {h^2}{8\pi^2I}l(l + 1),
    \label{TV}
\end{equation}

где $ I $ - момент инерции молекулы, а $ l $ - целое неотрицательное число. при $ l = 0 $ вращения не возбуждены. Характеристическая температура для вращения молекул определяется формулой:

\begin{equation}
    T_r = \frac {\epsilon_1}{k} = \frac {h^2}{4\pi^2Ik}
    \label{TV}
\end{equation}

При $ T >> T_r $ справедлива классическая теория, при $ T << T_r $ вращения не возбуждены и не оказывают влияния на теплоемкость. По той же причине не возбуждены вращения атомов одноатомарных газов. \\ [0.2cm]

\newpage

Вышеописанные соображения применимы к любым квантовым системам. Они показывают, что дискретность энергитических уровней несовместима с классической теоремой о равномерном распределении энергии по степеням свободы. Только при энергии теплового движения $ kT $ много больше разности энергий между высшими и низшими энергитическими уровнями, система ведет себя как классическая. Т.е. чем выше температура, тем лучше оправдывается классическая теорема о равномерном распределении энергии по степеням свободы.

\subsection {Квантовая теория теплоемкости Эйнштейна}